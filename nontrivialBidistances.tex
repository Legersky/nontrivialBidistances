\documentclass[a4paper, 11pt]{article}

\usepackage{cite}
\usepackage[english]{babel}
\usepackage[utf8]{inputenc}
% \usepackage[IL2]{fontenc}  
% \usepackage{amsmath,amsthm} 
% \usepackage{amsfonts}
\usepackage{amsmath, amsthm, amssymb, units, dsfont}

\usepackage[fixlanguage]{babelbib}
\selectbiblanguage{english}

\usepackage[pdfauthor={Jan Legersk\'y},
            pdfproducer={Jan Legersk\'y},
            pdfcreator={pdflatex},
            pdfencoding=unicode]{hyperref}
           
\usepackage{enumerate}
\usepackage[pdftex]{graphicx,color}


\newcommand{\komentar}[1]{\textcolor{red}{#1} \newline}

\newcommand{\trcomp}{$\triangle$-component}
\newcommand{\trcomps}{$\triangle$-components}
\newcommand{\cv}[1]{c_v^{(#1)}}

\newtheorem{thm}{Theorem}[section]
\newtheorem{lem}{Lemma}[section]
\newtheorem{cor}[lem]{Corollary}
\newtheorem*{rem}{Remark}

\theoremstyle{definition}
\newtheorem{defn}{Definition}[section]
\newtheorem{exmp}{Example}[section]
\newtheorem{conj}{Conjecture}

\begin{document}
\begin{defn}
\emph{A graph $G=(V,E)$ has a nontrivial bidistance} if the bigraph $(G,G)$ has a nontrivial bidistance with the trivial weight vector $w$, i.e., $w(e)=0$ for all $e\in E$.
\end{defn}


\section{Vertex which is not contained in $C_3\subset G$}
\begin{lem}
If $(G,G)$ is a bigraph, where $G$ is a simple graph, then $(d_G, -d_G)$ is a bidistance if and only if the minimum of $d_G$ occurs more than once in every cycle of $G$ and the same for the maximum.
\end{lem}
\begin{proof}
The maximum of $d_G$ corresponds to the minimum of $-d_G$ and vice versa. The claim follows by the definition of bidistance.
\end{proof}

\begin{cor}
\label{cor:binaryLabelling}
A bigraph $(G,G)$ has a nontrivial bidistance $(d_G, -d_G)$ iff there exists an edge labeling $\delta:E\rightarrow \{0,1\}$ such that $\delta(E)=\{0,1\}$ and for every cycle in $G$, neither 0 or 1 occurs exactly once.
\end{cor}
\begin{proof}
$\implies$: We may assume that $d_G(E)$ has at least one positive element, otherwise take $-d_G$. Set
\begin{align*}
\delta(e)&=1 \qquad \text{if } d_G(e)>0 \\
		&=0 \qquad \text{otherwise.}
\end{align*}
If we consider a cycle $C$ in $G$, then there are following possibilities:
\begin{enumerate}
	\item $\max_{e\in C} d_G(e) \leq 0 \implies \delta(e)=0$ for all $e \in C$.
	\item If $\max_{e\in C} d_G(e) > 0 \geq \min_{e\in C} d_G(e)$, then $\delta(e)=1$ for at least two distinct edges by the previous lemma and the same for $\delta(e)=0$.
	\item $\min_{e\in C} d_G(e) > 0 \implies \delta(e)=1$ for all $e \in C$.
\end{enumerate}
$\impliedby$: Take $d_G(e)=\delta(e)$ for all $e\in E$ if $\delta(\bar{e})=0$, or $d_G(e)=1-\delta(e)$ if $\delta(\bar{e})=1$, where $\bar{e}$ is the distinguished edge.
\end{proof}


\begin{rem}
Obviously, if a disconnected graph has at least two components which contain an edge, then it has a nontrivial bidistance. Also any graph with at least two edges and some vertex of degree one has a nontrivial bidistance.
\end{rem}

\begin{thm}
\label{thm:vertexNotInTriangle}
Let $G=(V,E)$ be a connected graph such that $|V|\geq 3$. If there is a vertex $v_0 \in V$ such that it is not contained in any triangle $C_3\subset G$, then $G$ allows a nontrivial bidistance. Moreover, $G$ has a non-rigid realization (informal proof...).
\end{thm}
\begin{proof}
Set
\begin{align*}
\delta(e)&=1 \qquad \text{if } v_0 \in e \\
		&=0 \qquad \text{otherwise.}
\end{align*} 
Now consider a cycle $C$ in $G$. If $v_0\notin C$, then $\delta(e)=0$ for all $e\in C$. If $v_0\in C$ then the label 1 occurs exactly twice (for the edges adjacent to $v_0$) and the label 0 at least twice as the length of C is at least four.

If there exists an edge which is not associated to $v_0$, then $\delta(E)=\{0,1\}$. Otherwise, the graph $G$ is a star, i.e., it has clearly a nontrivial bidistance.

To obtain a non-rigid realization, put all neighbours of $v_0$ to the same point $p$. This is allowed as they are non-adjacent. Now the vertex $v_0$ can rotate around $p$.
\end{proof}

\section{All vertices in some $C_3\subset G$}
Now we focus on graphs which do not satisfy the condition of Theorem~\ref{thm:vertexNotInTriangle}.
 

\begin{defn}
Let $G=(V,E)$ be a connected graph such that for all $v\in V$ there is some $C_3\subset G$ such that $v\in C_3$. We define a relation on $E\times E$ by 
$$e_1 \sim_{\!\!\bigtriangleup} e_2 \iff e_1=e_2 \text{ or } \exists\, C_3\subset G: e_1, e_2\in C_3\,.$$
Let $S_1, \dots, S_n$ be equivalence classes of the transitive closure of $\sim_{\!\!\bigtriangleup}$. The subgraph of $G$ with vertices $\{v\in V \colon \exists\, e\in S_i (v\in e)\}$ and edges $S_i$ is called a \emph{\trcomp{} (triangle-component)} if $|S_i|\geq 3$, and a \emph{connecting edge} otherwise.
\end{defn}

\begin{rem}
Every vertex is in some \trcomp{} as it is contained in some $C_3$. There might be \trcomps{} $T,T'$ such that $V_T \cap V_{T'}\neq \emptyset$. But always $E_T \cap E_{T'}= \emptyset$.
\end{rem}

\begin{rem}
From now on, we implicitly assume that $G=(V,E)$ is such that for all $v\in V$ there is some $C_3\subset G$ such that $v\in C_3$ whenever we use concept of \trcomps{} or connecting edges.
\end{rem}

\begin{lem}
\label{lem:bidistanceInTrcomp}
Let $d_G$ be a bidistance of a graph $G$. If $T=(V_T, E_T)$ is a \trcomp{} of $G$, then $d_G(e)=d_G(e')$ for all $e,e'\in E_T$.
\end{lem}
\begin{proof}
It is enough to prove that $e \sim_{\!\!\bigtriangleup} e'$ implies that $d_G(e)=d_G(e')$. That must be true, otherwise the maximum or minimum occurs only once in $C_3\subset G$ such that $e,e'\in C_3$.
\end{proof}





\begin{defn}
Let $G=(V,E)$ be a graph. A vertex $v\in V$ is called \emph{multiple vertex of multiplicity} $k$ if there exists exactly $k$ distinct \trcomps{} which contain $v$. A vertex $u\in V$ is called \emph{connecting vertex} if it is a multiple vertex or an endpoint of some connecting edge.
\end{defn}

\begin{lem}
If $T=(V_T, E_T)$ is a \trcomp{}, then $|E_T|\geq 2|V_T|-3$.
\label{lem:sizeTriangleComponent}
\end{lem}
\begin{proof}
If all vertices are of degree at least four, then $$2|E_T|=\sum_{v\in V_T} \deg(v) \geq 4|V_T|\,,$$ which implies $|E_T|\geq 2|V_T|\geq 2|V_T|-3$.

If there is a vertex $v\in V_T$ such that $\deg(v)<4$, we proceed by induction with the respect to the number of vertices. Obviously, the inequality holds for triangle. If the degree of $v$ is two, then by induction assumption
$$
|E_T|=|E_{T\setminus v}|+2\geq 2 |V_{T\\v}|-3 +2\geq 2 (|V_{T\setminus v}|+1)-3=2 |V_{T}|-3\,.
$$
We have two cases for $\deg(v)=3$. Firstly, $T\setminus v$ is a \trcomp{}. Now $$|E_T|=|E_{T\setminus v}|+3\geq 2 |V_{T\\v}|-3 +3\geq 2 (|V_{T\setminus v}|+1)-2\geq 2 |V_{T}|-3\,.$$ Secondly, $T\setminus v$ consists of two \trcomps{} $T_1$ and $T_2$. We have
\begin{align*}
|E_T|=|E_{T_1}|+|E_{T_1}|+3 &\geq (2 |V_{T_1}|-3)+(2 |V_{T_2}|-3) +3 \\
			&= 2(|V_{T_1}| +|V_{T_2}|)-3=2 |V_{T}|-3\,,
\end{align*}
where we use the fact that $|T_1 \cap T_2|=1$.
\end{proof}



\begin{lem}
\label{lem:noEdgeInSameComponent}
If $T=(V_T, E_T)$ is a \trcomp{} of  a Laman graph $G$, then $T$ is a Laman graph and there is no connecting edge with both vertices in $T$.
\end{lem}
\begin{proof}
The Laman conditions and Lemma~\ref{lem:sizeTriangleComponent} give $|E_T|= 2|V_T|-3$, hence $T$ is a Laman graph.

If there is a connecting edge with both vertices in $T$, then
$$
|E_{T\cup e}|=|E_T| +1 =2|V_T| -3+1\,,
$$
which contradicts that $G$ is Laman.
\end{proof}

\begin{lem}
\label{lem:threeEdgesOrVertexAndEdge}
Let $G$ be a Laman graph. If $T_1$ and $T_2$ are its \trcomps{}, then they can be connected by at most three connecting edges or by one multiple vertex with possibly one connecting edge. If there are three connecting edges, then they do not have a common vertex.
\end{lem}
\begin{proof}
Let $c_v$ be the number of connecting vertices, i.e., $c_v=|V_{T_1}\cap V_{T_2}|$, and $c_e$ be the number of connecting edges between $T_1$ and $T_2$. Using that $|E_{T_i}|= 2|V_{T_i}|-3$ and the Laman condition, we have
\begin{align*}
2(|V_{T_1}|+|V_{T_2}|)-6 +c_e&=|E_{T_1}|+|E_{T_2}|+c_e=|E_{H}| \\
&\leq 2|V_H| -3=2(|V_{T_1}|+|V_{T_2}|-c_v)-3\,,
\end{align*}
where subgraph $H$ is given by the union of $T_1$, $T_2$ and connecting edges. Hence $c_e+2c_v \leq 3$ which implies the first part of the statement. For the second one, assume that there are three connecting edges with the common vertex  $v\in T_2$. If $H'$ is a subgraph obtained by union of $T_1$, $v$ and the connecting vertices, then
$$
|E_{H'}|=|E_{T_1}|+3=2|V_T|=2(|V_T|+1)-2=2|V_{H'}|-2\,,
$$
which is a contradiction with the Laman condition.
\end{proof}

\begin{defn}
Let $T_1, \dots, T_n$ be \trcomps{} of a Laman graph $G=(V,E)$. Denote by $c_e$ the number of all connecting edges and by $\cv{k}$ the number of multiple vertices of multiplicity $k$.
\end{defn}

\begin{lem}
\label{lem:numCvCe}
If  $T_1, \dots, T_n$ are \trcomps{} of a Laman graph $G=(V,E)$, then
$$
3(n-1)=c_e + 2\sum_{k=2}^n \cv{k}(k-1)\,.
$$
\end{lem}
\begin{proof}
For every $T_i=(V_i,E_i)$ the equality $|E_i|=2|V_i|-3$ holds. For the number of vertices, we have
$$
|V|=\sum_{i=1}^n |V_i| - \sum_{k=2}^n \cv{k}(k-1)\,.
$$
Now we can obtain
\begin{align*}
c_e+\sum_{i=1}^n (2|V_i|-3)&=c_e + \sum_{i=1}^n |E_i|= |E|\\
	&=2|V|-3 = 2\sum_{i=1}^n |V_i| - 2\sum_{k=2}^n \cv{k}(k-1) -3\,.
\end{align*}
Hence
$$-3n + c_e= - 2\sum_{k=2}^n \cv{k}(k-1) -3\,,
$$
which gives the claim.
\end{proof}

\begin{lem}
\label{lem:twoConnectingVertices}
If $u,v\in V$ are two multiple vertices such that their multiplicities are $r$ and $s$, then
$$
r+s\leq n+1\,.
$$
\end{lem}
\begin{proof}
The vertex $u$ connects $r$ \trcomps{} which one of them can contain $v$. But there must be another $s-1$ \trcomps{} which contain $v$, otherwise there are two \trcomps{} that are connected by two vertices which contradicts Lemma~\ref{lem:threeEdgesOrVertexAndEdge}. Hence, there is at least $r+s-1$ distinct \trcomps{}.
\end{proof}

\begin{cor}
\label{cor:validCv}
Let $n\geq 2$ be the number of \trcomps{} in a Laman graph. The following statements hold:
\begin{enumerate}[i)]
	\item $\cv{n} \leq 1$.
	\item If $\cv{n} = 1$, then $\cv{k}=0$ for all $2\leq k<n$.
	\item If $n\geq 4$, then $\cv{n-1} \leq 1$.
	\item If $\cv{n-1} = 1$, then $\cv{k}=0$ for all $3\leq k<n-1$.
\end{enumerate}
\end{cor}
\begin{proof}
We always take two connecting vertices to get a contradiction with Lemma~\ref{lem:twoConnectingVertices}:
\begin{enumerate}[i)]
	\item $n+n >n +1$.
	\item $n +k>n+1$.
	\item $n-1+n-1>n+1$ for $n\geq 4$.
	\item $n-1+k>n+1$
\end{enumerate}
\end{proof}

\begin{thm}
\label{thm:componentWithNonadjacentVertices}
Let $G$ be a Laman graph with at least two \trcomps{}. If there is a \trcomp{} $T$ of $G$ such that no pair of its connecting vertices is adjacent, then there exists a nontrivial bidistance.  Moreover, $G$ has a non-rigid realization (informal proof...).
\end{thm}
\begin{proof}
We use Corollary~\ref{cor:binaryLabelling} for the proof. Set 
\begin{align*}
\delta(e)&=1 \qquad \text{if } e\in T \\
		&=0 \qquad \text{otherwise.}
\end{align*} 
All cycles going through $T$ contains 1 at least twice as there is no edge between connecting vertices. Also, there are at least two 0, if a cycle is not only in $T$, by Lemma~\ref{lem:noEdgeInSameComponent}. Obviously, $\delta(E)=\{0,1\}$ as there are at least two \trcomps{}.
\end{proof}

\begin{cor}
Let $T_1$ and $T_2$ be distinct \trcomps{} or connecting edges of a Laman graph $G$. If $T_i$ has only two connecting vertices $u$ and $v_i$ (or $T_i=uv_i$ if $T_i$ is a connecting edge) for $i\in\{1,2\}$, then  the graph $G$ has a nontrivial bidistance.  Moreover, $G$ has a non-rigid realization (informal proof...).
\end{cor}
\begin{proof}
We have $uv_1\notin E$ or $uv_2\notin E$ or $v_1v_2\notin E$, otherwise $T_1$ and $T_2$ coincide.
If $uv_1\notin E$ or $uv_2\notin E$, the statement follows from Lemma~\ref{thm:componentWithNonadjacentVertices}.

If $v_1v_2\notin E$, we set 
\begin{align*}
\delta(e)&=1 \qquad \text{if } e\in T_1\cup T_2 \\
		&=0 \qquad \text{otherwise.}
\end{align*} 
If a cycle goes through $T_1$ or $T_2$, then it must go through  $v_1, u$ and $v_2$. Therefore, the label 1 is there at least twice. At the same time, there are at least two labels 0 since  $v_1v_2\notin E$.
\end{proof}

\begin{lem}
\label{lem:numEdgesBetweenTwoParts}
Let $G=(V,E)$ be a Laman graph. If $H_1=(V_1,E_1)$ and $H_2=(V_2,E_2)$ are induced subgraphs of $G$ such that $V_1 \cap V_2=\emptyset$ and $V=V_1\cup V_2$, then the set $E_c=\{uv\in E | u\in H_1 \text{ and } v\in H_2\}$ contains at least three edges.
\end{lem}
\begin{proof}
Set $E'=E\setminus(E_1 \cup E_2)$. Now
$$
2(|V_1|+|V_2|)-3=2|V|-3=|E|=|E_1|+|E_2|+|E_c| \leq 2(|V_1|+|V_2|)-6 +|E_c|\,,
$$
which implies the statement.
\end{proof}

\begin{defn}
Let $G=(V,E)$ be a graph and $H=(W,F)$ be its subgraph. We say that the $H$ \emph{can be replaced by} a graph $H'=(W',F')$ in $G$ iff
$$
W'\cap V=\{v\in W | \exists\, u\in V \colon uv\in E\}\,.
$$
The graph obtained by the replacement is defined as
$$
G_{H\rightarrow H'}=\left((V\setminus W)\cup W', (E\setminus F)\cup F'\right)\,.
$$
\end{defn}

\begin{lem}
Let $L=(V_L, E_L)$ and $L'=(V_{L'}, E_{L'})$ be Laman graphs such that $L$ can be replaced by $L'$ in a graph $G=(V,E)$. If $G$ is a Laman graph, then $G_{L\rightarrow L'}=(V',E')$ is also a Laman graph. 
\end{lem}
\begin{proof}
The Laman condition on the whole graph holds:
$$
|E'|=|E|- |E_{L}|+ |E_{L'}|=2(|V|- |V_{L}|+ |V_{L'}|)-3+3-3=2|V'|-3\,.
$$
Suppose in contradiction that there exist a subgraph ${H'}=(V_{H'},E_{H'})$ of $G_{L\rightarrow L'}$ such that $|E_{H'}|> 2|V_{H'}|-3$. Set 
\begin{align*}
 V_{in}&=V_{H'}\cap V_{L'}, &E_{in}&=E_{H'}\cap E_{L'}, \\
 V_{out}&=V_{H'} \setminus V_{in}, & E_{out}&=E_{H'} \setminus E_{in}
\end{align*}

Since $(V_{in}, E_{in})$ is a subgraph of the Laman graph $L'$, we have
$$
2|V_{in}|-3 +|E_{out}|\geq  |E_{in}|+|E_{out}|=|E_{H'}|> 2|V_{{H'}}|-3=2|V_{in}|+2|V_{out}|-3\,.
$$
Hence, $|E_{out}|>2|V_{out}|$. Take the subgraph $H=(V_L \cup V_{out}, E_L \cup E_{out})$ of $G$. Using that $|E_L|=2|V_L|-3$, we obtain
$$|E_H|=|E_L|+|E_{out}|>2|V_L|-3+2|V_{out}|=2|V_H|-3\,,$$
which is  a contradiction with $G$ being a  Laman graph.
\end{proof}

\begin{cor}
\label{cor:replaceTrcompByLaman}
Let $T=(V_T,E_T)$ be a \trcomp{} of a graph $G=(V,E)$ with connecting vertices $v_1, \dots, v_r$. If $L=(V_L,E_L)$ is a Laman graph with the set of vertices $V_L=\{v_1, \dots, v_r\}$ and $T$ is a Laman graph, then the graph $G_{T\rightarrow L}$ is a Laman graph iff $G$ is a Laman graph.
\end{cor}

\begin{lem}
\label{lem:trcompToEdge}
Let $T=(V_T,E_T)$ be a \trcomp{} of a Laman graph $G=(V,E)$ with only two connecting vertices $u$ and $v$. If $uv\in E$, then $G$ has a nontrivial bidistance iff $G_{T\rightarrow uv}$ has a nontrivial bidistance.
\end{lem}
\begin{proof}
Note that $G_{T\rightarrow uv}$ is a subgraph of $G$. Let $\delta_G$ and $\delta_{G_{T\rightarrow uv}}$ by the edge labelings which correspond to the bidistances of $G$ and $G_{T\rightarrow uv}$ according to Lemma~\ref{cor:binaryLabelling}.

$\implies$: The restriction of $\delta_G$ to  $G_{T\rightarrow uv}$ gives the labeling $\delta_{G_{T\rightarrow uv}}$.

$\impliedby$: Define $\delta_G(e)=\delta_{G_{T\rightarrow uv}}(e)$ for all $e\in G_{T\rightarrow uv}$ and $\delta_G(e)=\delta_{G_{T\rightarrow uv}}(uv)$ for all $e\in T$. Since for every cycle going through $uv$, the label $\delta_{G_{T\rightarrow uv}}(uv)$ occurs at least twice, it must occur at least twice in arbitrary cycle through $T$.
\end{proof}




\begin{thm}
\label{thm:cutToTwoParts}
Let $G=(V,E)$ be a Laman graph. Let  $T_1, \dots, T_r$ be some of the \trcomps{} of $G$ which have only two connecting vertices, namely $u_1, v_1, \dots u_r,v_r$, where  $T_i=(V_i,E_i)$ and $u_i, v_i\in V_i$. Let  $u_{r+1}v_{r+1}, \dots u_s v_s$ be some of the connecting edges of $G$. Let $G'$ be the graph obtained from $G$ by replacing $T_i$, resp. $u_iv_i$, by $(\{u_i, v_i\}, \emptyset)$ for all $i\in \{1, \dots, s\}$.
% Set 
%\begin{align*}
%V_{ncon}&=\left\{v\in V_1 \cup \dots \cup V_r | v\notin \{u_1, \dots u_r\}\cup \{v_1, \dots v_r\} \right\} \qquad \text{and} \\
%E_{con}&=E_1 \cup \dots \cup E_r \cup \{u_{r+1}v_{r+1}, \dots u_s v_s\}\,.
%\end{align*}
If $\{u_1, \dots, u_s\}$ are in the different connected component(s) of the graph $G'$ than $\{v_1, \dots, v_s\}$, then there exists a nontrivial bidistance.  Moreover, $G$ has a non-rigid realization (informal proof...).
\end{thm}
\begin{proof}
We remark that there might be no \trcomps{} ($r=0$) or no connecting edges ($s=r$), but always $s\geq 3$ by Lemma~\ref{lem:numEdgesBetweenTwoParts} and Corollary~\ref{cor:replaceTrcompByLaman}.

If there is $T_i$ such that $u_i v_i \notin E$, then there exists a nontrivial bidistance by Theorem~\ref{thm:componentWithNonadjacentVertices}.
Otherwise, we may assume that there are only connecting edges  $u_1 v_1, \dots u_r v_r$ instead of \trcomps{} $T_1, \dots, T_r$ according to Lemma~\ref{lem:trcompToEdge}.

The idea is following: since the graph $G'\subset G$ is disconnected, every cycle in $G$ which does not stay in a connected component of $G'$ must go through the given  connecting edges at least twice.

We may assume that all $\{u_1, \dots, u_s\}$ are in the same connected component $M$, because if we can construct a bidistance in such case, then we may just forgot all connecting edges associated to the elements of $\{u_1, \dots, u_s\}$ which are not in the given connected component in the general case. By the same argument, we may assume that $\{v_1, \dots, v_s\}$ are in one connected component $N$. Thus $G'$ consists of only two connected components $M$ and $N$, otherwise $G$ is not connected. In fact, $M$ and $N$ are induced subgraphs of $G$ linked by the connecting edges $\{u_1v_1, \dots, u_sv_s\}$.

Set $\delta(u_i v_i)=1$ for all $i\in \{1, \dots, s\}$ and $\delta(e)=0$ otherwise. Let $C$ be a cycle in $G$. If $C\subset M$ or  $C\subset N$, then all labels are 0. Otherwise, $C$ contains some $u_iv_i$ and $u_jv_j$ such that $u_iv_i\neq u_jv_j$. Hence, the label 1 is in $C$ at least twice. If $u_i\neq u_j$ and $v_j\neq v_j$, then there must be at least one edge in $M\cap C$ and at least one edge in $N\cap C$.  If $u_i=u_j$, then $v_i$ and $v_j$ are non-adjacent, otherwise $u_iv_i \sim_{\!\!\bigtriangleup} u_jv_j$. Thus, there are at least two edges in $C\cap N$. The case $v_i=v_j$ is analogous. Therefore, the label 0 is in $C$ at least twice.
\end{proof}

\begin{thm}
Let $G$ be a Laman graph with \trcomps{} $T_1,\dots, T_n$. If $n=1$, then there is no nontrivial bidistance. If $n=2,3,4$ or $5$, then $G$ allows nontrivial bidistance.
\end{thm}
\begin{proof}[Proof (sketch)]
If $n=1$, then all edges have the same bidistance by Lemma~\ref{lem:bidistanceInTrcomp}, i.e. 0.

For $n=2,3,4$ or $5$, we consider all valid combinations of $c_e$ and $\cv{k}$ according to Lemma~\ref{lem:numCvCe} and Corollary~\ref{cor:validCv}. We construct all possible valid connections of $n$ \trcomps{} for given $c_e$ and $\cv{k}$ by using Lemma~\ref{lem:threeEdgesOrVertexAndEdge}. For most of them, the existence of a bidistance follows from Theorem~\ref{thm:componentWithNonadjacentVertices} and \ref{thm:cutToTwoParts}, otherwise we find a nontrivial bidistance ad hoc.
\end{proof}

\begin{conj}
Let $G$ be a Laman graph with \trcomps{} $T_1,\dots, T_n$. There exists a nontrivial bidistance iff $n>1$.
\end{conj}


\end{document}
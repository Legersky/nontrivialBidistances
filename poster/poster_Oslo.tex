\documentclass[a0paper, portrait, margin=0mm, innermargin=15mm, blockverticalspace=15mm, colspace=15mm, subcolspace=-5mm]{tikzposter}
\tikzposterlatexaffectionproofoff



\usepackage{tabulary}
\usepackage{caption}

%\usepackage{cite}
%\usepackage[english]{babel}
%\usepackage[utf8]{inputenc}
\usepackage{amssymb, amsmath}

\newtheorem{thm}{Theorem}
%\newtheorem*{lem}{Lemma}
%\newtheorem*{cor}{Corollary}
%\newtheorem*{rem}{Remark}
%
%\theoremstyle{definition}
\newtheorem{defn}{Definition}
%\newtheorem*{exmp}{Example}
%\newtheorem*{conj}{Conjecture}

\title{Nonrigid realizations of Laman graphs} 
\author{Jan Legersk\'y \\ ESR 7, Research Institute for Symbolic Computation, JKU Linz, Austria} 
%\institute{Czech Technical University in Prague}


\definecolorstyle{Czech} {
\definecolor{colorOne}{HTML}{34888C}%116699
\definecolor{colorTwo}{HTML}{C1E1DC}
\definecolor{colorThree}{HTML}{FFF2BE}
%\definecolor{colorOne}{HTML}{34888C}%116699
%\definecolor{colorTwo}{HTML}{7CAA2D}
%\definecolor{colorThree}{HTML}{F5E356}
%\definecolor{colorOne}{HTML}{4F6457}%116699
%\definecolor{colorTwo}{HTML}{D9B44A}
%\definecolor{colorThree}{HTML}{ACD0C0}
%\definecolor{colorOne}{HTML}{0D5078}%116699
%\definecolor{colorTwo}{HTML}{A2C4D9}
%\definecolor{colorThree}{HTML}{FCF0AD}
}{
     % Background Colors
    \colorlet{backgroundcolor}{colorTwo}
    \colorlet{framecolor}{colorThree}
    % Title Colors
    \colorlet{titlebgcolor}{colorOne}
    \colorlet{titlefgcolor}{white}
    % Block Colors
    \colorlet{blocktitlebgcolor}{white}
    \colorlet{blocktitlefgcolor}{colorOne}
    \colorlet{blockbodybgcolor}{white}
    \colorlet{blockbodyfgcolor}{black}
    % Innerblock Colors
    \colorlet{innerblocktitlebgcolor}{white}
    \colorlet{innerblocktitlefgcolor}{black}
    \colorlet{innerblockbodybgcolor}{colorThree}
    \colorlet{innerblockbodyfgcolor}{black}
    % Note colors
    \colorlet{notefgcolor}{black}
    \colorlet{notebgcolor}{colorThree}
    \colorlet{notefrcolor}{colorThree}
}


\usebackgroundstyle{Default} %Rays
\usetitlestyle{Filled}
\usecolorstyle{Czech}

\useblockstyle[bodyoffsety=12mm]{Slide}
\usenotestyle{Default}



%\settitle{ \centering \vbox{
%%\@titlegraphic\\[\TP@titlegraphictotitledistance] 
%\centering
%\color{titlefgcolor} {\bfseries \Huge \sc \@title \par}
%\vspace*{1em}
%{\huge \@author \par}% \vspace*{1em} {\LARGE \@institute}
%}}

\newcommand{\RR}{\mathbb{R}}
\newcommand{\trcomp}{$\triangle$-component}
\newcommand{\trcomps}{$\triangle$-components}
\newcommand{\cv}[1]{c_v^{(#1)}}


\begin{document}
\maketitle[titletoblockverticalspace=15mm]
\begin{columns} 
\column{0.5}
\block{Abstract}{
TO DO
}

\block{Rigidity and Laman graphs}{
\begin{defn}
Let $G=(V_G,E_G)$ be a simple graph with a set of vertices $V_G$ and a set of edges $E_G$. 
\begin{itemize}
	\item An \emph{embedding} of $G$ is a map $\varepsilon:V_G\rightarrow \RR^2$. 
	\item  Let $\lambda:E_G\rightarrow \RR_+$ be an edge labelling of $G$. An embedding $\varepsilon$ is \emph{compatible with} $\lambda$ iff $\forall uv=e\in E_G \colon ||\varepsilon(u)-\varepsilon(v)||^2=\lambda(e)$.
	%\item A labelled graph $(G,\lambda)$ is \emph{realizable} iff it has a compatible embedding.
	%\item Two embeddings $\varepsilon_1$ and $\varepsilon_2$ are equivalent iff there exists a direct Euclidean isometry $\sigma$ of $\RR^2$ such that $\varepsilon_1=\sigma \circ\varepsilon_2$.
	\item A labelled graph $(G,\lambda)$ is \emph{rigid} iff the number of embeddings of $G$ compatible with $\lambda$ up rotations and translations is finite and nonzero. %to equivalence.
	\item A graph $G$ is \emph{generically rigid} iff $(G,\lambda)$ is rigid for a generic labelling $\lambda$.
	\item A graph $G$ has a \emph{nonrigid realization OR EMBEDDING/LABELLING?} iff there exists a labelling $\lambda$ such that the number of embeddings of $G$ compatible with $\lambda$ is up rotations and translations is infinite.
\end{itemize} 
\end{defn}
\begin{thm}
REFERENCE A graph $G$ is generically rigid iff it is so called \emph{Laman graph} that means
\begin{itemize}
	\item $|E_G|=2|V_G|-3$ and
	\item $|E_H|\leq 2|V_H|-3$ for every subgraph $H$ of $G$. 
\end{itemize}  
\end{thm}
SOME PICTURE OF NONRIGID REALIZATION OF THREE PRISM GRAPH
}
\block{Bidistance}{
\begin{defn}
\emph{A bidistance} of a graph $G$ is an edge labelling $\delta:E_G\rightarrow \{0,1\}$ such that for every cycle in $G$, neither 0 or 1 occurs exactly once. A bidistance is \emph{nontrivial} iff $\delta(E_G)=\{0,1\}$.
\end{defn}
NOTE ABOUT GENERAL BIDISTANCE IN BIGRAPH???
\begin{thm}
If a graph $G$ has a nonrigid realization, then it has a nontrivial bidistance.
\end{thm}
PICTURE OF BIDISTANCE OF THREE PRISM GRAPH
}

\block{Easy case}{
\begin{thm}
\label{thm:vertexNotInTriangle}
Let $G=(V,E)$ be a connected graph such that $|V|\geq 3$. If there is a vertex $v_0 \in V$ such that it is not contained in any triangle $C_3\subset G$, then $G$ allows a nontrivial bidistance. Moreover, $G$ has a nonrigid realization.
\end{thm}
PICTURE OF K3,3?
}



\column{0.5}
\block{\trcomps{}}{
\begin{defn}
Let $G$ be a connected graph such that all vertices are in some triangle $C_3\subset G$. We define a relation on $E_G\times E_G$ by 
$$e_1 \sim_{\!\!\bigtriangleup} e_2 \iff  \exists\, C_3\subset G: e_1, e_2\in C_3\,.$$
Let $E_1, \dots, E_n$ be equivalence classes of the reflexive and transitive closure of $\sim_{\!\!\bigtriangleup}$ on $E_G$. The subgraph $T_i=(V_i,E_i)$ of $G$, where  $V_i=\{v\in V | \exists\, e\in E_i \colon v\in e\}$, is called a \emph{\trcomp{} (triangle-component)} iff $|E_i|\geq 3$, and a \emph{connecting edge} otherwise.
\end{defn}
\begin{defn}
A vertex $v\in V$ is called \emph{multiple vertex of multiplicity} $k$ if there exists exactly $k$ distinct \trcomps{} which contain $v$. A vertex $u\in V$ is called \emph{connecting vertex} if it is a multiple vertex or an endpoint of some connecting edge.
\end{defn}
\begin{thm}
Let $T$ be a \trcomp{} of a graph $G$. The following statements hold:
\begin{itemize}
	\item $|E_T|\geq 2|V_T|-3$.
	\item If $\delta$ is a bidistance of $G$, then for all $e,e'\in E_T$, $\delta(e)=\delta(e')$.
	\item The \trcomp{} $T$ has no nonrigid realization.
	\item The \trcomp{} $T$ is a Laman graph iff it is a 2-tree. (A graph $G$ is called \emph{a 2-tree}, if either $G$ are only two adjacent vertices, or there exist a vertex $v\in V_G$ of degree two, whose neighbours are adjacent and $G\setminus v$ is a 2-tree.)
\end{itemize}
\end{thm}
\begin{thm}
Let $G$ be a Laman graph with $n$ \trcomp{}.  The following statements hold:
\begin{itemize}
	\item Every \trcomp{} is a planar Laman graph.
	\item There is no connecting edge with both endpoints in the same \trcomp{}.
	\item Any two \trcomps{} can be connected by at most three connecting edges or by one multiple vertex with possibly one connecting edge. If there are three connecting edges, then all three cannot have a common vertex.
	\item If $c_e$ is the number of all connecting edges in $G$ and $\cv{k}$ is the number of multiple vertices of multiplicity $k$ in $G$, then
$$
3(n-1)=c_e + 2\sum_{k=2}^n \cv{k}(k-1)\,.
$$
\end{itemize}
\end{thm}
}

\block{Sufficient conditions}{
\begin{thm}
A Laman graph $G$ with $n$ \trcomps{}, $n\geq 2$, has a nonrigid realization and nontrivial bidistance if some of the following conditions holds:
\begin{itemize}
	\item The number of all connecting edges $c_e$ is at least $n$.
	\item There exists a \trcomp{} in $G$ whose connecting vertices are pairwise nonadjacent.
	\item There are two distinct \trcomps{} which have only two connecting vertices and they are connected by one of them.
	\item There is a \trcomp{} connected only with one multiple vertex and one connecting edge.
	\item There are three \trcomps{} which are pairwise connected by a multiple vertex.
	\item 
\end{itemize}
\end{thm}
}

\end{columns}
%\center KM FJFI \v{C}VUT, Trojanova 13, 120 00 Praha 2, Czech Republic,  \url{jan.legersky@gmail.com}
\end{document}



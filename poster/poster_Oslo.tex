\documentclass[24pt, a0paper, portrait, margin=0mm, innermargin=15mm, blockverticalspace=15mm, colspace=15mm, subcolspace=-5mm]{tikzposter}
\tikzposterlatexaffectionproofoff



\usepackage{tabulary}
\usepackage{caption}

%\usepackage{cite}
%\usepackage[english]{babel}
%\usepackage[utf8]{inputenc}
\usepackage{amssymb}

\newtheorem{thm}{Theorem}
%\newtheorem*{lem}{Lemma}
%\newtheorem*{cor}{Corollary}
%\newtheorem*{rem}{Remark}
%
%\theoremstyle{definition}
\newtheorem{defn}{Definition}
%\newtheorem*{exmp}{Example}
%\newtheorem*{conj}{Conjecture}

\title{Nonrigid realizations of Laman graphs} 
\author{Jan Legersk\'y, ESR 7, Research Institute for Symbolic Computation, JKU Linz} 
%\institute{Czech Technical University in Prague}


\definecolorstyle{Czech} {
\definecolor{colorOne}{HTML}{34888C}%116699
\definecolor{colorTwo}{HTML}{C1E1DC}
\definecolor{colorThree}{HTML}{FFF2BE}
%\definecolor{colorOne}{HTML}{34888C}%116699
%\definecolor{colorTwo}{HTML}{7CAA2D}
%\definecolor{colorThree}{HTML}{F5E356}
%\definecolor{colorOne}{HTML}{4F6457}%116699
%\definecolor{colorTwo}{HTML}{D9B44A}
%\definecolor{colorThree}{HTML}{ACD0C0}
%\definecolor{colorOne}{HTML}{0D5078}%116699
%\definecolor{colorTwo}{HTML}{A2C4D9}
%\definecolor{colorThree}{HTML}{FCF0AD}
}{
     % Background Colors
    \colorlet{backgroundcolor}{colorTwo}
    \colorlet{framecolor}{colorThree}
    % Title Colors
    \colorlet{titlebgcolor}{colorOne}
    \colorlet{titlefgcolor}{white}
    % Block Colors
    \colorlet{blocktitlebgcolor}{white}
    \colorlet{blocktitlefgcolor}{colorOne}
    \colorlet{blockbodybgcolor}{white}
    \colorlet{blockbodyfgcolor}{black}
    % Innerblock Colors
    \colorlet{innerblocktitlebgcolor}{white}
    \colorlet{innerblocktitlefgcolor}{black}
    \colorlet{innerblockbodybgcolor}{colorThree}
    \colorlet{innerblockbodyfgcolor}{black}
    % Note colors
    \colorlet{notefgcolor}{black}
    \colorlet{notebgcolor}{colorThree}
    \colorlet{notefrcolor}{colorThree}
}


\usebackgroundstyle{Default} %Rays
\usetitlestyle{Filled}
\usecolorstyle{Czech}

\useblockstyle[bodyoffsety=12mm]{Slide}
\usenotestyle{Default}



\settitle{ \centering \vbox{
%\@titlegraphic\\[\TP@titlegraphictotitledistance] 
\centering
\color{titlefgcolor} {\bfseries \Huge \sc \@title \par}
\vspace*{1em}
{\huge \@author \par}% \vspace*{1em} {\LARGE \@institute}
}}

\newcommand{\RR}{\mathbb{R}}


\begin{document}
\maketitle[titletoblockverticalspace=15mm]
\begin{columns} 
\column{0.5}
\block{Abstract}{
asdsjcnksnvcskjdn
}

\block{Rigidity and Laman graphs}{
\begin{defn}
Let $G=(V,E)$ be a simple graph with a set of vertices $V$ and a set of edges $E$. 
\begin{itemize}
	\item An \emph{embedding} of $G$ is a map $\varepsilon:V\rightarrow \RR^2$. 
	\item  Let $\lambda:E\rightarrow \RR_+$ be an edge labelling of $G$. An embedding $\varepsilon$ is \emph{compatible with} $\lambda$ iff $\forall uv=e\in E \colon ||\varepsilon(u)-\varepsilon(v)||^2=\lambda(e)$.
	%\item A labelled graph $(G,\lambda)$ is \emph{realizable} iff it has a compatible embedding.
	%\item Two embeddings $\varepsilon_1$ and $\varepsilon_2$ are equivalent iff there exists a direct Euclidean isometry $\sigma$ of $\RR^2$ such that $\varepsilon_1=\sigma \circ\varepsilon_2$.
	\item A labelled graph $(G,\lambda)$ is \emph{rigid} iff the number of embeddings of $G$ compatible with $\lambda$ up rotations and translations is finite and nonzero. %to equivalence.
	\item A graph $G$ is \emph{generically rigid} iff $(G,\lambda)$ is rigid for a generic labelling $\lambda$.
	\item A graph $G$ has a \emph{nonrigid realization} iff there exists a labelling $\lambda$ such that the number of embeddings of $G$ compatible with $\lambda$ is up rotations and translations is infinite.
\end{itemize} 
\end{defn}
\begin{thm}
A graph $G$ is generically rigid iff it is so called \emph{Laman graph} that means
\begin{itemize}
	\item $|E|=2|V|-3$ and
	\item $|E_H|\leq 2|V_H|-3$ for every subgraph $H=(V_H,E_H)$ of $G$. 
\end{itemize}  
\end{thm}
}



%\note[targetoffsetx=0cm,targetoffsety=11pt,innersep=0.5cm,angle=-90, width=\colwidth]{\url{jan.legersky@gmail.com}, \url{github.com/Legersky/ParallelAddition}}
%%\block{}{}
%
%\column{0.6}
%\input{ewm.tex}
%
%
%
%\begin{subcolumns}
%\subcolumn{0.43}
%\input{phase1.tex}
%\input{phase2img.tex}
%\subcolumn{0.57}
%\input{phase1img.tex}
%\input{phase2.tex}
%\end{subcolumns}
%
%\input{phase1convergence.tex}
%
%%\begin{subcolumns}
%%\subcolumn{0.5}
%%
%%
%%\subcolumn{0.5}
%%
%%
%%\end{subcolumns}
%
%
%\input{phase2convergence.tex}
%\input{examples.tex}
%
%
%\block{}{
%C.~Frougny, E.~Pelantov\'a, and M.~Svobodov\'a, \emph{Parallel addition in
%  non-standard numeration systems}, Theoret. Comput. Sci. \textbf{412} (2011),
%  5714--5727.
%
%%C.~Frougny, E.~Pelantov{\'a}, and M.~Svobodov{\'a}, \emph{Minimal digit sets
%%  for parallel addition in non-standard numeration systems}, J. Integer Seq.
%%  \textbf{16} (2013), 36.
%}


\end{columns}
%\center KM FJFI \v{C}VUT, Trojanova 13, 120 00 Praha 2, Czech Republic,  \url{jan.legersky@gmail.com}
\end{document}


